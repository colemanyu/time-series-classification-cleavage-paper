% \textbf{Background:} 
MicroRNAs (miRNAs) are small non-coding RNAs (ncRNAs) that regulate gene expression at the post-transcriptional level, thereby playing essential roles in diverse biological processes.
% Such as development and differentiation.
%
The biogenesis of miRNAs requires Dicer to cleave at specific sites on the precursor miRNAs (pre-miRNAs).
%
Several machine learning approaches have been proposed to predict whether an input sequence contains a cleavage site. However, they rely heavily on complex feature engineering or opaque deep neural networks. 
It results in a lack of generalizability and a long running time.
% Additionally, the probabilities of the base pairs in the predicted secondary structure have been ignored in the classification.
There is a need for an alternative modeling paradigm that is accurate, fast, and simple.

% \textbf{Results:} 
We propose an approach to frame the task as a multivariate time series classification problem. 
Nine encoding methods have been proposed to convert the RNA sequence into a time series.
The predicted secondary structure is also converted to a time series.
We also leverage the probabilities of the base pairs in the predicted secondary structure.
Computational experiments demonstrate that our proposed method can achieve better or comparable results in terms of using a simpler, more intuitive model and less computational time.
It achieves 3.7X to 28.8X speedup.
Through perturbation experiments, we found that regions close to the center of pre-miRNAs are essential for predicting human Dicer cleavage sites.

% \textbf{Conclusion:} 
% Our proposed scheme enables us to approach this problem in a novel way. 
By transforming the RNA sequence and its secondary structure information into a multivariate time series and utilizing simple, state-of-the-art time series classifiers, we achieved comparable or even superior performance in a simpler and faster manner.
% We introduced the use of probabilities of base pairs in the classification. 
% The analysis of the importance of the subsequences suggests that the regions close to the center of the pre-miRNA are essential for this problem.

Code is available at: \url{https://github.com/colemanyu/time-series-classification-cleavage}.