\section{Conclusions} \label{sec:conclusions}
We proposed a simple, fast, and accurate multivariate time series classification-based method, termed MTSCCleav, for predicting human dicer cleavage sites by transforming nucleotide sequences and the secondary structures into time series.
Base-pair probability sequences of the secondary structures have also been used in the method.
MTSCCleav consists of three parts: time series encoding, time series transformation, and classification.
ROCKET-based methods were used for time series transformation.
Ridge Classifier was used for classification.
For the computational experiments, we evaluated nine time series encoding methods with five time series transformation methods.
For the three datasets: 5p-arm, 3p-arm, and multi-class, MTSCCleav outperformed the SOTA methods in all five evaluation metrics for the 5p-arm and multi-class datasets, and four of the metrics for the 3p-arm dataset.
In terms of computational efficiency, MTSCCleav with the optimal setting achieves an average 3.7X to 27.0X speedup over the SOTA method on the three datasets.
With the use of a less accurate but faster time series transformation method, MTSCCleav achieves an average speedup of 16.1X to 28.8X, respectively.
We analyzed the subsequence importance of the input multivariate time series.
The results show that subsequences near the center of the pre-miRNA sequences are more important.
It agrees with the findings from the previous work.

Overall, this study demonstrates that time series analysis provides a powerful alternative to conventional modeling in the context of RNA processing. 
This framework may be extended to other RNA-processing tasks.

%%%
%%%
%%%
