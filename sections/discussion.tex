\section{Discussion} \label{sec:discussion}
% More accurate, faster
Our results show that MTSCCleav outperforms the SOTA, DiCleave, in both accuracy and efficiency.
The channel ablation study reveals that the involvement of the time series derived from the secondary structure can improve accuracy.
It suggests the importance of RNA folding in Dicer processing.
Additionally, we found that the base-pair probability sequence of the secondary structure can also enhance accuracy.
To the best of our knowledge, it is a novel application of the base-pair probability sequence.
We can make use of the probability sequence either by involving it in the time series encoding of the complementary strand or as a standalone channel in the multivariate time series.
Experiments show that using the probability sequence as an additional channel can enhance accuracy more than involving it in the encoding.
It is likely because keeping it as an additional channel can preserve more information, of both the probability sequence itself and the encoded complementary strand.

% Discuss classifier
Out of the three datasets, the best classifier is ROCKET.
The ranking of the five classifiers by performance, starting from the best, is as follows: ROCKET, Hydra, MiniROCKET, MultiROCKET-Hydra, MultiROCKET.
It indicates that the features created from the pooling operations of the activation map, which are only in MultiROCKET but not in MiniROCKET, confuse the final classifier.
They are mean of positive values (MPV), mean of indices of positive values (MIPV) and longest stretch of positive values (LSPV)~\cite{MultiRocketMultiplePooling2022}.
In contrast, the pooling operator, which is only present in ROCKET but not in MiniROCKET, enhances the classification performance. 
It is maximum (MAX).
% Discuss encoding methods, why, any overall observations
For the encoding methods, we have the following observations.
Fixed-length grouped channel mappings outperform variable-length counterparts with one exception in the multi-class dataset, likely because fixed-length schemes better preserve the original positional information of nucleotides within the sequence.
Global cumulative methods, which compute cumulative mappings starting from the beginning of the sequence, consistently yield better performance than local cumulative methods, with one exception in the 3p dataset.
It suggests that the upstream information of the cleavage pattern plays a critical role in identifying cleavage sites.
Cumulative-based encodings perform better than single-value mappings, with one exception in the 3p dataset, suggesting that the accumulated nucleotide signal is more informative for cleavage site prediction than the local or isolated presence of nucleotides.
% In the 3p dataset, grouped fixed-length is better than its counterpart. 
In the 5p dataset, encoding in two channels rather than one channel appears to worsen the result.
This suggests that the same grouping method used for the 5p-arm dataset cannot be applied to the 3p-arm dataset.

% Using few data
One limitation of DiCleave is overfitting during training because of the relatively small size of the dataset~\cite{DiCleaveDeepLearning2024}.
DiCleave is a deep learning-based method.
Deep learning models typically require a large amount of training data to generalize effectively.
They are data-hungry.
In contrast, MTSCCleav leverages ROCKET-based methods for the classification.
They rely on random convolutional feature extraction followed by a linear classifier.
A ridge classifier was used in this study.
Ridge classifiers are less data-hungry compared to deep learning methods due to their use of L2 regularization and the simplicity of their linear model nature.
It allows ROCKET-based classifiers, and hence MTSCCleav, to maintain strong predictive performance even in settings with a relatively small dataset size.

% Explain feature importance, why, any overall observations
The subsequence importance of MTSCCleav reveals some connections between RNA secondary structure and human dicer cleavage site prediction.
The perturbation experiment shows that the leading part of 5p-arm and the tailing part of 3p-arm are important for the classification.
These parts are close to the center of the RNA secondary structure of pre-miRNA.
It indicates that the center region is more crucial for human dicer cleavage site prediction.
It agrees with the previous study~\cite{ReCGBMGradientBoostingbased2021}.
%%%








