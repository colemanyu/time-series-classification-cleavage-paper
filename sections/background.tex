\section{Background} \label{sec:background}
% Biological background
% DNA (a segment in DNA = a gene) -> RNA -> Protein (Functional unit)
% miRNA
One of the most important theories in molecular biology is the central dogma.  It depicts the flow of genetic information~\cite{CampbellBiology2020, MolecularBiologyCell2022}.
Proteins are the functional units. And the information stored in DNA is used to make them up.
This is not a one-step process. It involves transcription and translation.
Genes (segments of DNA encoding proteins) in DNA are used as templates for messenger RNAs (mRNAs) synthesis. This synthesis process is called transcription.
An mRNA acts as a set of instructions to assemble a chain of amino acids, which form a linear polypeptide. This construction process is called translation.
This chain is not yet functional. To become biologically active, this chain is folded into a specific three-dimensional structure, a proper configuration. This process is called protein folding.
And this folded polypeptide is called a functional protein, or a protein in short.

This whole process is remarkably similar to the process of running a computer program on a computer. The source code does not function by itself. The source code turns into assembly code (less human-readable code) first, and finally into an executable that does the work the programmer intends. 
In this analogy, DNA acts as the whole source code, in which genes refer to the functions in the code (segments of the code). mRNA is the assembly code. And the final product proteins are the running executable.

These mRNAs are called ``coding RNAs'' because they code for proteins.
There are other genes in which the final product is the RNA molecule itself. 
They are called non-coding RNAs (ncRNAs).
They do functions by themselves, such as regulating gene expression.
Gene expression refers to the process by which the genetic information is transcribed into RNA, which may be translated into protein or ncRNAs.
Two types of small ncRNAs are particularly important. 
They are microRNAs (miRNAs) and small interfering RNAs (siRNAs).
Their discovery was recognized with the 2006 Nobel Prize in Physiology or Medicine\footnote{The Nobel Prize in Physiology or Medicine 2006 - NobelPrize.org: \\\href{https://www.nobelprize.org/prizes/medicine/2006/summary/}{https://www.nobelprize.org/prizes/medicine/2006/summary/} (Accessed on: 2025-06-13).}, awarded for work completed only eight years prior~\cite{CampbellBiology2020}.
They play essential roles in gene expression regulation.
In this manuscript, we focus on miRNAs. They are small RNAs with a length of about 22 nt.
% miRNAs are important.
They regulate gene expression post-transcriptionally~\cite{MicroRNAsGenomicsBiogenesis2004}. 
A miRNA can regulate the expression of several proteins.
They are essential in diverse biological processes, such as development, differentiation, and disease~\cite{MicroRNAGeneExpression2005, RoleMicroRNAGenes2005}.
Hence, it is of great value to understand the formation or the biogenesis of miRNAs.
It involves the processing of primary miRNAs (pri-miRNAs).
A predicted secondary structure of a pri-miRNA's sequence is shown in Figure~\ref{fig:hsa-let-7a-1_ss}.

\begin{figure}[htbp]
\centerline{\includegraphics[width=0.6\columnwidth]{figures/hsa-let-7a-1_ss.svg.pptx.pdf}}
\caption{
Secondary structure of the pri-miRNA ``hsa-let-7a-1''\protect\footnotemark.
We denote the sequence as $S$.
Experimental evidence suggests that the two deviated mature miRNAs are $UGA \cdots GUU$ and $CUA \cdots UUC$.
They are $S[6:27]$ and $S[57:77]$.
Since $S[6:27]$ ($S[57:77]$) is near the 5' (3') end, we call it ``5p (3p) mature miRNA''. 
The starting and ending indices of these two subsequences are indicated in \textcolor{red}{\textbf{bold and red}} in the figure.
It suggests that the two cleavage sites are the two bonds immediately after the 27\textsuperscript{th} nucleotide and before the 57\textsuperscript{th} nucleotide.
The two scissors indicate the two cleavage sites.
The color intensity of the nodes reflects their base pair probability in this predicted secondary structure configuration.
The unpaired nodes are uncolored.
The raw figure is generated by RNAfold web server\protect\footnotemark.
}
\label{fig:hsa-let-7a-1_ss}
\end{figure}
% Coleman
% https://tex.stackexchange.com/questions/10181/using-footnote-in-a-figures-caption
% https://tex.stackexchange.com/questions/536265/using-two-footnote-in-a-figures-caption
\addtocounter{footnote}{-1}
\footnotetext{Its miRBase entry: \href{https://mirbase.org/hairpin/MI0000060}{https://mirbase.org/hairpin/MI0000060}. (Accessed on: 2025-06-12).}
\addtocounter{footnote}{1}
\footnotetext{RNAfold web server: \href{http://rna.tbi.univie.ac.at/cgi-bin/RNAWebSuite/RNAfold.cgi}{http://rna.tbi.univie.ac.at/cgi-bin/RNAWebSuite/RNAfold.cgi}. (Accessed on: 2025-06-12).
The figure is viewed in forna. This view option can be chosen on the previous website.}

Recall that a pri-miRNA contains a hairpin loop, also called a stem loop. It is located on the bottom left part of Figure~\ref{fig:hsa-let-7a-1_ss}.
A microprocessor complex comprising Drosa and DCGR8 cleaves the pri-miRNA to form a precursor miRNA (pre-miRNA) inside the nucleus. 
The stem-loop is still preserved, but the two arms become shorter.
The two arms refer to the two ends of the sequence.
The sequence or strand has the end-to-end chemical orientation.
One end is called 5' end, and the other is called 3' end.
After that,  the pri-miRNA is transported by Exportin 5 from the nucleus to the cytoplasm (i.e., outside the nucleus).
It is further cleaved by an enzyme called Dicer~\cite{MicroRNAMaturationStepwise2002}. 
The Dicer cleaves the stem-loop from the two arms at the two cleavage sites, shown as the two scissors in figure~\ref{fig:hsa-let-7a-1_ss}.
We call the bond between two nucleotides along the strand that is cleaved by Dicer the Dicer cleavage site.  
The stem-loop is removed.
It results in a short double-stranded miRNA molecule.
% of 18-25 nucleotides.
Furthermore, these molecules may be subjected to additional trimming. Some nucleotides are removed from the 5' (5p) and 3' (3'p) ends.

One strand of the double-stranded miRNA molecule is loaded into an RNA-induced silencing complex (RISC).
This loaded strand guides the RISC to the target mRNA to silence it, and it results in gene silencing.
This loaded strand is called the guide strand or mature miRNA.
The other stand, which is called the passenger strand, is usually degraded.
Note that both miRNA single strands, resulting from the unwinding of the double-stranded miRNA molecule, can become the guide strand. 
For example, “hsa-let-7a-1” has two guide strands or mature miRNA products.

Dicer plays an important role in the biogenesis of miRNAs.
It would be of great benefit to understand how Dicer selects cleavage sites from the neighborhood information near the cleavage sites.
The neighborhood information refers to both the sequence and the secondary structure information.
Studies~\cite{LoopPositionShRNAs2012, ComprehensiveAnalysisPrecursor2012, StructuralDeterminantsRNA2007} revealed that the secondary structures of the sequence are essential for cleavage site determination.
Hence, to predict or classify whether a subsequence, extracted from pri-miRNAs, contains a cleavage site, we need to make use of both the sequence and secondary structure information.

\subsection{Related work}
Cleavage site prediction is not only defined on human dicer. Other examples include the calpain~\cite{CalpainCleavagePrediction2011} and caspasesl~\cite{SVMbasedPredictionCaspase2006}, which are proteases that cleave proteins. Most of the studies about cleavage site prediction focus on protein cleavage sites. A wide range of computational methods has been applied on this topic, including support vector machine~\cite{SVMbasedPredictionCaspase2006}, deep neural networks~\cite{PrecisePredictionCalpain2019}.
In this manuscript, we focus on human dicer cleavage sites.
We review the works on human dicer cleavage sites here.
PHDcleav employed support vector machines (SVM) leveraging sequence and structure-based features~\cite{PHDcleavSVMBased2013}.
LBSizeCleav improved upon it by considering the loop and buldge lengths~\cite{LBSizeCleavImprovedSupport2016}.
\cite{ReCGBMGradientBoostingbased2021} proposed a ensemble learning approach, used gradient boosting machine for better accuracy.
\cite{DiCleaveDeepLearning2024} developed a deep learning model namely DiCleave. This model used an autoencoder to learn the secondary structure embeddings of pre-miRNA.


% General cleavage problem
% Human dicer cleavage
% Limitation
%% different scheme for different cleavage problem
%% hard coding for the feature
%% Time series release this constraint


% We need to prepare a dataset that contains the positive samples and the negative samples. Positive (negative) samples refer to the strings that (do not) contain a cleavage site. 

% Contributions
In summary, our contributions are shown as follow.
\begin{enumerate}
    \item To the best of our knowledge, we are the first to frame the prediction of the cleavage sites, not only Dicer cleavage sites, as a multivariate time series classification problem. By using time series modeling, it bridges bioinformatics and time series which allows us to use state-of-the-art time series classification algorithms. Besides, the visualization power of time series allows us understand the data more intuitively.
    \item We propose of making use the base-pair probabilities in the predicted secondary structure in the prediction. To our surprise, this information has been ignored in the existing works. We leverage this base-pair probabilities to our novel transformation scheme for RNA sequence and its complementary sequence.
    \item We conduct extensive experiments on different transformation methods and convolution based classification methods.
\end{enumerate}

%%%




