\begin{table*}[htbp]
    \centering
      % Overfull \hbox (x pt too wide) 
      % https://blog.csdn.net/weixin_46777569/article/details/126260667
      % \resizebox{\columnwidth}{!}{
      \begin{tabular}{c | c}
         \hline
         Sequence & \makecell{
         Secondary Structure \\ 
         (In Dot-bracket notation)
         } \\
         \hline  
         \makecell{
         1 UGGGA\textcolor{red}{\textbf{UGAGGUAGUAGGUUGUAUAGUU}} 27\\
         28 UUAGGGUCACACCCACCACUGGGAGAU 54\\
         55 AA\textcolor{red}{\textbf{CUAUACAAUCUACUGUCUUUC}}CUA 80
         } &
         \makecell{
         1 (((((\textcolor{red}{\textbf{.(((((((((((((((((((((}} 27\\
         28 UUAGGGUCACACCCACCACUGGGAGAU 54\\
         55 ))\textcolor{red}{\textbf{)))))))))))))))))))))}}))) 80
         } \\
         \hline
         Base-pair probabilities sequence (the first 10 bases) & \\
         \hline
         \makecell{
         1 (0.549, 0.946, 0.987, 0.987, 0.904) 5 \\
         6 (\textcolor{red}{\textbf{0.000}}, 0.841, 0.974, 0.981, 0.890) 10
         } & \\
         \hline
    \end{tabular}
    % }
    % https://tex.stackexchange.com/questions/531/what-is-the-best-way-to-use-quotation-mark-glyphs
    \caption{The whole sequence of ``hsa-let-7a-1'' and its predicted secondary structure by RNAfold.
    % We have numbered each line with the starting and ending positions.
    The corresponding positions of the two mature miRNAs and the probability of the unpaired ``U'' are highlighted in \textcolor{red}{\textbf{bold}}.}
    \label{tab:hsa-let-7a-1}
\end{table*}