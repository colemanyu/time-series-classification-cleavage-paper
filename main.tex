%Version 3.1 December 2024
% See section 11 of the User Manual for version history
%
%%%%%%%%%%%%%%%%%%%%%%%%%%%%%%%%%%%%%%%%%%%%%%%%%%%%%%%%%%%%%%%%%%%%%%
%%                                                                 %%
%% Please do not use \input{...} to include other tex files.       %%
%% Submit your LaTeX manuscript as one .tex document.              %%
%%                                                                 %%
%% All additional figures and files should be attached             %%
%% separately and not embedded in the \TeX\ document itself.       %%
%%                                                                 %%
%%%%%%%%%%%%%%%%%%%%%%%%%%%%%%%%%%%%%%%%%%%%%%%%%%%%%%%%%%%%%%%%%%%%%

%%\documentclass[referee,sn-basic]{sn-jnl}% referee option is meant for double line spacing

%%=======================================================%%
%% to print line numbers in the margin use lineno option %%
%%=======================================================%%

%%\documentclass[lineno,pdflatex,sn-basic]{sn-jnl}% Basic Springer Nature Reference Style/Chemistry Reference Style

%%=========================================================================================%%
%% the documentclass is set to pdflatex as default. You can delete it if not appropriate.  %%
%%=========================================================================================%%

%%\documentclass[sn-basic]{sn-jnl}% Basic Springer Nature Reference Style/Chemistry Reference Style

%%Note: the following reference styles support Namedate and Numbered referencing. By default the style follows the most common style. To switch between the options you can add or remove “Numbered” in the optional parenthesis. 
%%The option is available for: sn-basic.bst, sn-chicago.bst%  
 
%%\documentclass[pdflatex,sn-nature]{sn-jnl}% Style for submissions to Nature Portfolio journals
%%\documentclass[pdflatex,sn-basic]{sn-jnl}% Basic Springer Nature Reference Style/Chemistry Reference Style
\documentclass[pdflatex,sn-mathphys-num]{sn-jnl}% Math and Physical Sciences Numbered Reference Style
%%\documentclass[pdflatex,sn-mathphys-ay]{sn-jnl}% Math and Physical Sciences Author Year Reference Style
%%\documentclass[pdflatex,sn-aps]{sn-jnl}% American Physical Society (APS) Reference Style
%%\documentclass[pdflatex,sn-vancouver-num]{sn-jnl}% Vancouver Numbered Reference Style
%%\documentclass[pdflatex,sn-vancouver-ay]{sn-jnl}% Vancouver Author Year Reference Style
%%\documentclass[pdflatex,sn-apa]{sn-jnl}% APA Reference Style
%%\documentclass[pdflatex,sn-chicago]{sn-jnl}% Chicago-based Humanities Reference Style

%%%% Standard Packages
%%<additional latex packages if required can be included here>

\usepackage{graphicx}%
\usepackage{multirow}%
\usepackage{amsmath,amssymb,amsfonts}%
\usepackage{amsthm}%
% Coleman
% Fix "Font shape `U/rsfs/m/n' in size <8.43146> not available"
% \usepackage{mathrsfs}%
\usepackage[title]{appendix}%
\usepackage{xcolor}%
\usepackage{textcomp}%
\usepackage{manyfoot}%
\usepackage{booktabs}%
\usepackage{algorithm}%
\usepackage{algorithmicx}%
\usepackage{algpseudocode}%
\usepackage{listings}%

% Coleman
\usepackage{makecell}
% sub figure
\usepackage{subcaption} % "Package caption Warning"
%%%%

%%%%%=============================================================================%%%%
%%%%  Remarks: This template is provided to aid authors with the preparation
%%%%  of original research articles intended for submission to journals published 
%%%%  by Springer Nature. The guidance has been prepared in partnership with 
%%%%  production teams to conform to Springer Nature technical requirements. 
%%%%  Editorial and presentation requirements differ among journal portfolios and 
%%%%  research disciplines. You may find sections in this template are irrelevant 
%%%%  to your work and are empowered to omit any such section if allowed by the 
%%%%  journal you intend to submit to. The submission guidelines and policies 
%%%%  of the journal take precedence. A detailed User Manual is available in the 
%%%%  template package for technical guidance.
%%%%%=============================================================================%%%%

%% as per the requirement new theorem styles can be included as shown below
\theoremstyle{thmstyleone}%
\newtheorem{theorem}{Theorem}%  meant for continuous numbers
%%\newtheorem{theorem}{Theorem}[section]% meant for sectionwise numbers
%% optional argument [theorem] produces theorem numbering sequence instead of independent numbers for Proposition
\newtheorem{proposition}[theorem]{Proposition}% 
%%\newtheorem{proposition}{Proposition}% to get separate numbers for theorem and proposition etc.

\theoremstyle{thmstyletwo}%
\newtheorem{example}{Example}%
\newtheorem{remark}{Remark}%

\theoremstyle{thmstylethree}%
\newtheorem{definition}{Definition}%

\raggedbottom
%%\unnumbered% uncomment this for unnumbered level heads

% Coleman
% Multi-file
% https://www.overleaf.com/learn/latex/Multi-file_LaTeX_projects
\usepackage{blindtext}
\usepackage{subfiles} % Best loaded last in the preamble

% Adding line numbers
\usepackage{lineno}
\linenumbers
\begin{document}
% Coleman
% \title[short title]{full title}
\title[MTSCCleav]{MTSCCleav: a Multivariate Time Series Classification (MTSC)-based method for predicting human Dicer cleavage sites }

%%=============================================================%%
%% GivenName	-> \fnm{Joergen W.}
%% Particle	-> \spfx{van der} -> surname prefix
%% FamilyName	-> \sur{Ploeg}
%% Suffix	-> \sfx{IV}
%% \author*[1,2]{\fnm{Joergen W.} \spfx{van der} \sur{Ploeg} 
%%  \sfx{IV}}\email{iauthor@gmail.com}
%%=============================================================%%

\author*[1,2]{\fnm{First} \sur{Author}}\email{iauthor@gmail.com}

\author[2,3]{\fnm{Second} \sur{Author}}\email{iiauthor@gmail.com}
\equalcont{These authors contributed equally to this work.}

\author[1,2]{\fnm{Third} \sur{Author}}\email{iiiauthor@gmail.com}
\equalcont{These authors contributed equally to this work.}

\affil*[1]{\orgdiv{Department}, \orgname{Organization}, \orgaddress{\street{Street}, \city{City}, \postcode{100190}, \state{State}, \country{Country}}}

\affil[2]{\orgdiv{Department}, \orgname{Organization}, \orgaddress{\street{Street}, \city{City}, \postcode{10587}, \state{State}, \country{Country}}}

\affil[3]{\orgdiv{Department}, \orgname{Organization}, \orgaddress{\street{Street}, \city{City}, \postcode{610101}, \state{State}, \country{Country}}}

% \author*[1]{\fnm{Coleman} \sur{Yu}}\email{cyu@kuicr.kyoto-u.ac.jp}

% \author[2]{\fnm{Raymond Chi-Wing} \sur{Wong}}\email{raywong@cse.ust.hk}

% \author[1]{\fnm{Tomoya} \sur{Mori}}\email{tmori@kuicr.kyoto-u.ac.jp}

% \author[1]{\fnm{Tatsuya} \sur{Akutsu}}\email{takutsu@kuicr.kyoto-u.ac.jp}


% \affil*[1]{\orgdiv{Bioinformatics Center, Institute for Chemical Research}, \orgname{Kyoto University}, \orgaddress{Uji, Kyoto 611-0011, Japan}}

% \affil[2]{\orgdiv{Department of Computer
% Science and Engineering}, \orgname{The Hong Kong University of Science and Technology}, \orgaddress{Clear Water Bay, Kowloon, Hong Kong}}

%%==================================%%
%% Sample for unstructured abstract %%
%%==================================%%

%%================================%%
%% Sample for structured abstract %%
%%================================%%
\abstract{
\textbf{Background:} 
MicroRNAs (miRNAs) are small non-coding RNAs (ncRNAs) that regulate gene expression at the post-transcriptional level, thereby playing essential roles in diverse biological processes.
% Such as development and differentiation.
The biogenesis of miRNAs requires dicer to cleave at specific sites on the precursor miRNAs (pre-miRNAs).
Several machine learning approaches have been proposed to predict whether an input sequence contains a cleavage site. However, they rely heavily on complex feature engineering or opaque deep neural networks. 
It results in a lack of generalizability and a long running time.
% Additionally, the probabilities of the base pairs in the predicted secondary structure have been ignored in the classification.
There is a need for an alternative modeling paradigm that is accurate, fast, and simple.

\textbf{Results:} 
We proposed a novel approach to frame the task as a multivariate time series classification problem. 
Various encoding methods have been proposed to convert the sequence and the predicted secondary structure into a time series.
We also leveraged the probabilities of the base pairs in the predicted secondary structure.
Computational experiments demonstrate that our proposed method can achieve better or comparable results using a simpler, more intuitive model and less computational time.
It achieves 3.7X $\sim$ 28.8X speedup.
Through perturbation experiments, we found that regions close to the center of pre-miRNAs are essential for predicting human dicer cleavage sites.

\textbf{Conclusion:} 
% Our proposed scheme enables us to approach this problem in a novel way. 
By transforming the RNA sequence and its secondary structure information into a time series and utilizing simple, state-of-the-art time series classifiers, we achieved comparable or even superior performance in a simpler and faster manner.
% We introduced the use of probabilities of base pairs in the classification. 
% The analysis of the importance of the subsequences suggests that the regions close to the center of the pre-miRNA are essential for this problem.
Code is available at: \href{https://github.com/cyuab/time-series-classification-cleavage}{https://github.com/cyuab/time-series-classification-cleavage}.
}
\keywords{miRNA, Dicer Cleavage Site, Genomic signal processing (GSP), (Multivariate) time Series Classification (MTSC, TSC)}
%%%
%%%
%%%

%%\pacs[JEL Classification]{D8, H51}

%%\pacs[MSC Classification]{35A01, 65L10, 65L12, 65L20, 65L70}

\maketitle

% Testing for bibliography~\cite{DiCleaveDeepLearning2024}.
\subfile{sections/background} % 1500 words, 2 pages
\subfile{sections/methods} % 1500 words
\subfile{sections/results} % 1000 ~ 1500 words
\subfile{sections/discussion} % 500 words
\subfile{sections/conclusions} % 500 words
\appendix
% \subfile{sections/appendix}

%%===========================================================================================%%
%% If you are submitting to one of the Nature Portfolio journals, using the eJP submission   %%
%% system, please include the references within the manuscript file itself. You may do this  %%
%% by copying the reference list from your .bbl file, paste it into the main manuscript .tex %%
%% file, and delete the associated \verb+\bibliography+ commands.                            %%
%%===========================================================================================%%

\bibliography{main}% common bib file
%% if required, the content of .bbl file can be included here once bbl is generated
%%\input sn-article.bbl

\end{document}
