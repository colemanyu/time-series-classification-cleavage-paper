\documentclass{ieeeaccess}
% Coleman: Libraries
% -------------------------------------------
% https://www.overleaf.com/learn/latex/Multi-file_LaTeX_projects
\usepackage{blindtext}
\usepackage{subfiles}
% Fix outline for References
\usepackage{etoolbox}
% -------------------------------------------
\usepackage{cite}
\usepackage{amsmath,amssymb,amsfonts}
\usepackage{algorithmic}
\usepackage{graphicx}
\usepackage{textcomp}

\usepackage{bm}
\makeatletter
\AtBeginDocument{\DeclareMathVersion{bold}
\SetSymbolFont{operators}{bold}{T1}{times}{b}{n}
\SetSymbolFont{NewLetters}{bold}{T1}{times}{b}{it}
\SetMathAlphabet{\mathrm}{bold}{T1}{times}{b}{n}
\SetMathAlphabet{\mathit}{bold}{T1}{times}{b}{it}
\SetMathAlphabet{\mathbf}{bold}{T1}{times}{b}{n}
\SetMathAlphabet{\mathtt}{bold}{OT1}{pcr}{b}{n}
\SetSymbolFont{symbols}{bold}{OMS}{cmsy}{b}{n}
\renewcommand\boldmath{\@nomath\boldmath\mathversion{bold}}}
\makeatother

\def\BibTeX{{\rm B\kern-.05em{\sc i\kern-.025em b}\kern-.08em
    T\kern-.1667em\lower.7ex\hbox{E}\kern-.125emX}}

% Coleman: Libraries
% -------------------------------------------
% Table
\usepackage{makecell}
% sub figure
% \usepackage{subcaption} % It results "Package caption Warning".

\definecolor{teal}{rgb}{0.0, 0.5, 0.5}

\usepackage{multirow}
\usepackage{booktabs}%

% \usepackage{subcaption}
\usepackage[caption=false]{subfig}

\usepackage{hyperref}
\usepackage{bookmark}
\usepackage{xurl}
\usepackage{hyperref} % Load this after xurl

% Fix outline for References
\makeatletter
% Coleman: Fix Reference Outline Link
% We patch \addcontentsline instead of \section*.
% This searches for the command that creates the bookmark and puts the anchor right before it.
\patchcmd{\thebibliography}
    {\addcontentsline}
    {\phantomsection\addcontentsline}
    {}{}
\makeatother
% -------------------------------------------
%Your document starts from here ___________________________________________________
\begin{document}
\history{Date of publication xxxx 00, 0000, date of current version xxxx 00, 0000.}
\doi{10.1109/ACCESS.2024.0429000}

\title{MTSCCleav: a Multivariate Time Series Classification (MTSC)-based Method for Predicting Human Dicer Cleavage Sites}
\author{\uppercase{Coleman Yu}\authorrefmark{1},
\uppercase{Raymond Chi-Wing Wong}\authorrefmark{2}, and \uppercase{Tatsuya Akutsu}\authorrefmark{3}, \IEEEmembership{Senior Member, IEEE}}

\address[1]{Bioinformatics Center, Institute for Chemical Research, Kyoto University, Japan (e-mail: cyu@kuicr.kyoto-u.ac.jp)}
\address[2]{Department of Computer Science and Engineering, The Hong Kong University of Science and Technology, Hong Kong (e-mail: raywong@cse.ust.hk)}
\address[3]{Bioinformatics Center, Institute for Chemical Research, Kyoto University, Japan (e-mail: takutsu@kuicr.kyoto-u.ac.jp)}
\tfootnote{The work of Tatsuya Akutsu was supported in part by Japan Society for the Promotion of Science (JSPS), Japan, under Grant 22H00532 and Grant 22K19830.}

\markboth
{Author \headeretal: Preparation of Papers for IEEE TRANSACTIONS and JOURNALS}
{Author \headeretal: Preparation of Papers for IEEE TRANSACTIONS and JOURNALS}

\corresp{Corresponding author: Coleman Yu (e-mail: cyu@kuicr.kyoto-u.ac.jp).}


\begin{abstract}
\subfile{sections/00-abstract}
\end{abstract}

\begin{keywords}
miRNA, Dicer cleavage site, genomic signal processing (GSP), (multivariate) time series classification (MTSC, TSC)
\end{keywords}

\titlepgskip=-21pt

\maketitle

\subfile{sections/01-background} % 1500 words, 2 pages
\subfile{sections/02-methods} % 1500 words
\subfile{sections/03-results} % 1000 ~ 1500 words
\subfile{sections/04-discussion} % 500 words
\subfile{sections/05-conclusion} % 500 words

% Coleman: thebibliography
% -------------------------------------------
\bibliographystyle{ieeetr}
\bibliography{access}
% -------------------------------------------

% Coleman: Libraries
% -------------------------------------------
\bookmarksetup{startatroot}
% -------------------------------------------
% Coleman: IEEEbiography
% -------------------------------------------
% Coleman: Outline
\phantomsection
\begin{IEEEbiography}[{\includegraphics[width=1in,height=1.25in,clip,keepaspectratio]{author-coleman-yu.jpg}}]{COLEMAN YU} received the 
BSc degree in Pure Physics, with minor programs in Mathematics, and Information Technology, in 2014
and the MPhil degree in Technology Leadership and Entrepreneurship, hosted in the Department of Computer Science and Engineering, in 2016
from The Hong Kong University of Science and Technology.

He is currently pursuing the PhD degree in Informatics with the Bioinformatics Center, Institute for Chemical Research, Kyoto University.
His research interests include data mining, particularly in time series, and bioinformatics.
\end{IEEEbiography}

% Coleman: Outline
\phantomsection
\begin{IEEEbiography}[{\includegraphics[width=1in,height=1.25in,clip,keepaspectratio]{author-raymond-wong.jpg}}]{RAYMOND CHI-WING WONG} 
Raymond Chi-Wing Wong is a Professor in Computer Science and Engineering (CSE) of The Hong Kong University of Science and Technology (HKUST). He is currently the associate head of Department of Computer Science and Engineering (CSE) and the director of Undergraduate Research Opportunities Program (UROP). He was the associate director of the Data Science \& Technology (DSCT) program (from 2019 to 2021), the director of the Risk Management and Business Intelligence (RMBI) program (from 2017 to 2019), the director of the Computer Engineering (CPEG) program (from 2014 to 2016) and the associate director of the Computer Engineering (CPEG) program (from 2012 to 2014). He received the BSc, MPhil and PhD degrees in Computer Science and Engineering in the Chinese University of Hong Kong (CUHK) in 2002, 2004 and 2008, respectively. In 2004-2005, he worked as a research and development assistant under an R\&D project funded by ITF and a local industrial company called Lifewood. He received 43 awards. He published 135 conference papers (e.g., SIGMOD, SIGKDD, VLDB, ICDE and ICDM), 50 journal/chapter papers (e.g., TODS, DAMI, TKDE, VLDB journal and TKDD) and 1 book. He reviewed papers from conferences and journals related to data mining and database, including VLDB conference, SIGMOD, TODS, VLDB Journal, TKDE, TKDD, ICDE, SIGKDD, ICDM, DAMI, DaWaK, PAKDD, EDBT and IJDWM. He is a program committee member of conferences, including SIGMOD, VLDB, ICDE, KDD, ICDM and SDM, and a referee of journals, including TODS, VLDBJ, TKDE, TKDD, DAMI and KAIS. His research interests include database and data mining.
\end{IEEEbiography}

% Coleman: Outline
\phantomsection
\begin{IEEEbiography}[{\includegraphics[width=1in,height=1.25in,clip,keepaspectratio]{author-tatsuya-akutsu.jpg}}]{TATSUYA AKUSTSU} (Senior Member, IEEE) received the B.E. and M.E. degrees
in aeronautics and the D.E. degree in information
engineering from the University of Tokyo, 1984, 1986, and 1989,
respectively.
He has been a professor in the Bioinformatics
Center, Institute for Chemical Research, Kyoto University since 2001.
His research interests include bioinformatics, complex networks,
neural networks. and discrete algorithms.
\end{IEEEbiography}
% -------------------------------------------

% \newpage

\EOD

\end{document}
